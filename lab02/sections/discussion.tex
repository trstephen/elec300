\section{Discussion}\label{sec:discussion}
A phase-lag circuit was built of an identical topology to that in Figure 1.
The frequency of a 1V peak-to-peak input signal was varied over ~12kHz in 1kHz increments. 
The input and output voltage and phase were measured using the in-lab oscilloscope.
The recorded data can be seen in Table 1.

In our pre-lab, we simulated the phase and amplitude response of the phase-lag system.
As can be seen by comparing Figures XXX and XXY, the simulated results closely matched experimentally generated values.
Both phase plots show a minimum phase of -55\si{\degree} occurring at ~3kHz. 
The capacitors used had an error of $\pm$20$\%$.
This might account for some of the difference between the simulated and experimental data.

A phase-lag circuit receives its name from the characteristics exhibited by the circuit's transfer function. 
The output voltage phase is always less than the input voltage phase, resulting in a constantly "lagging" phase.