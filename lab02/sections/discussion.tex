\clearpage		% make sure images from results don't interrupt section
\section{Discussion}\label{sec:discussion}
The phase-lag circuit, shown in Fig.~\ref{fig:schematic}, was built and analyzed.
The frequency of a 1V peak-to-peak input signal was varied over ~12kHz in 1kHz increments, and the input and output voltage and phase measured using the oscilloscope.
The recorded data can be seen in Table~\ref{table:data}.

In our pre-lab, we simulated the phase and amplitude response of the phase-lag system.
As can be seen in Fig.~\ref{fig:freq-response}, the simulated results closely matched experimentally generated values.
The phase plots in Fig.~\ref{subfig:phase-response} both demonstrate a minimum phase of -55\si{\degree} occurring at ~3kHz.

The measured phase response is slightly less negative than the simulated response, likely due to the ESR of the capacitors.

A phase-lag circuit receives its name from the characteristics exhibited by the circuit's transfer function.
The output voltage's phase is shifted negatively by the circuit, resulting in a lagging phase.
