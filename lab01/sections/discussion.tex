\section{Discussion}\label{sec:discussion}
% vcvs
\subsection{VCVS}
\textit{1.3 Does the slope of the straight line agree with the value of K in \eqref{eq:vcvs}}

Yes, the predicted value of 2 is reflected in the slope in Fig. \ref{fig:vcvs-graph} for $-4.2\si{\volt}<V_i<4.5\si{\volt}$.

Response is capped at expected values on positive side but is lower than expected on negative side. Why?

\textit{1.3 Determine the internal resistance of the op amp}

When we attempted to estimate the internal resistance using a 150\si{\ohm} load ($R_L$ in Fig. \ref{fig:schematics}\subref{fig:vcvs}), we were only receiving a drop in voltage of approximately 6\si{\micro\volt} (peak-to-peak), which implied an internal resistance of approximately 500\si{\micro\ohm}: this was considered to be incorrect. Replacing the 150\si{\ohm} load with a 47\si{\ohm} load, we experienced a more plausible result. With a load of 47\si{\ohm}, the voltage across the load was measured to be 788.5\si{\milli\volt}, which means the source has an estimated internal resistance of approximately 39.1\si{\ohm} (see~\eqref{eq:internal-resistance}).

\begin{equation*}
I_o	= \frac{V_o}{R_L} = \frac{0.7885\si{\volt}}{47\si{\ohm}} = 16\si{\milli\ampere}
\end{equation*}
\begin{equation*}
V_{internal}		= V_i - V_L = 1.4135\si{\volt} - 0.7885\si{\volt} = 625\si{\milli\volt}
\end{equation*}
\begin{equation}
\label{eq:internal-resistance}
R_{internal}	= \frac{V_{internal}}{I_o} = \frac{625\si{\milli\volt}}{16\si{\milli\ampere}} = 39.1\si{\ohm}
\end{equation}

\subsection{VCCS}
\textit{2.3 Compare the measured range of linear output to the expected range}

Evaluating \eqref{eq:max-vccs} gives $-5\si{\volt} V_i < 5\si{\volt}$, whereas we observed a linear region between approximately $-4.0\si{\volt}<V_i<4.2\si{\volt}$. 
When the edge of the linear region was reached, rather than hard limiting the current experience soft limiting where $G$ dropped to \num{328e-6}.
The asymmetric maxima are most likely the result of a mismatched input impedance.
\SI{47}{\ohm} was used to compensate for $R_{internal}$, which differs from the expected value in \eqref{eq:internal-resistance}.

\textit{2.4 What is the effect on $I_o$ when $R_L: \SI{1}{\kilo\ohm} \to \SI{0}{\ohm}$?}

When shorting the output carrying 1\si{\milli\ampere} when measured under a 1\si{\kilo\ohm} load, the measured current dropped to 0.951\si{\milli\ampere}.

\large{\textit{Add a internal resistance calculation. Not sure how to do this eq'n.}}

Measuring the voltage between the inverting and non-inverting gives a reading of 0.97\si{\milli\volt}. In comparison to other voltages within the circuit and considering the immense input impedance, the current into each pin is indeed nearly non-existent and certainly negligible.

\textit{2.5 Confirm the validity of the ideal op amp assumptions}
We measured \[V^+ - V^1 = \SI{0.97}{\milli\volt}\], which is very close to the ideal op amp input difference of 0 V.
